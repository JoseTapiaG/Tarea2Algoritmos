
%----------------------------------------------------------------------------------------
%	PACKAGES AND OTHER DOCUMENT CONFIGURATIONS
%----------------------------------------------------------------------------------------

\documentclass[
12pt, % Main document font size
letterpaper, % Paper type, use 'letterpaper' for US Letter paper
oneside, % One page layout (no page indentation)
%twoside, % Two page layout (page indentation for binding and different headers)
headinclude,footinclude, % Extra spacing for the header and footer
BCOR5mm, % Binding correction
]{scrartcl}

\input{structure.tex} % Include the structure.tex file which specified the document structure and layout


%----------------------------------------------------------------------------------------

\begin{document}

%----------------------------------------------------------------------------------------
%	HEADERS
%----------------------------------------------------------------------------------------

\renewcommand{\sectionmark}[1]{\markright{\spacedlowsmallcaps{#1}}} % The header for all pages (oneside) or for even pages (twoside)
%\renewcommand{\subsectionmark}[1]{\markright{\thesubsection~#1}} % Uncomment when using the twoside option - this modifies the header on odd pages
\lehead{\mbox{\llap{\small\thepage\kern1em\color{halfgray} \vline}\color{halfgray}\hspace{0.5em}\rightmark\hfil}} % The header style

\pagestyle{scrheadings} % Enable the headers specified in this block

%----------------------------------------------------------------------------------------
%   TITLE PAGE
%----------------------------------------------------------------------------------------
\begin{flushright}
\includegraphics[width=8cm]{dcc.jpg}
\end{flushright}	
\vskip50mm
\begin{center}
% TITLE
\huge\textbf{Estudio de algoritmos en memoria secundaria}
\vskip2mm
% SUBTITLE (optional)
\LARGE\textit{Tarea numero 2}
\vskip5mm
% nombre del curso
\Large\textbf{Diseño y análisis de Algoritmos}
\normalsize
\end{center}
\vskip70mm

\begin{tabular}{lll}
\textbf{\textit{Autora}} & : & Romina Romero Oropesa \\
						& & José Manuel Tapia\\
\textbf{\textit{Profesor}} & : & Pablo Barceló\\
\textbf{\textit{Curso}} & : & CC4102 \\ 
\textbf{\textit{Fecha}} & : & Viernes 6 de Junio, 2014
\end{tabular}

\newpage
%----------------------------------------------------------------------------------------
%	TABLE OF CONTENTS & LISTS OF FIGURES AND TABLES
%----------------------------------------------------------------------------------------

\setcounter{tocdepth}{2} % Set the depth of the table of contents to show sections and subsections only

\tableofcontents % Print the table of contents

\newpage 
%----------------------------------------------------------------------------------------
%	Hipótesis
%----------------------------------------------------------------------------------------

\section{Hipótesis}

Ejemplo de lista:
%ejemplo lista
\begin{description}
\item[Word] Definition
\item[Concept] Explanation
\item[Idea] Text
	\begin{itemize}
	\item cosa1
	\item cosa2
	\end{itemize}
\end{description}
 
%----------------------------------------------------------------------------------------
%	Diseño experimental
%----------------------------------------------------------------------------------------
\newpage
\section{Diseño experimental}
Ejemplo de código
%ejemplo codigp
\begin{lstlisting}[language=Java]
class Hello{
	static public void main(String[] args){
		System.out.println("Hello world!\n");
	}
}
\end{lstlisting}

\newpage

\section{Resultados}



\newpage

\section{Análisis e Interpretación}
Ejemplo ecuación:
%ejemplo ecuacion
\begin{equation}
\cos^3 \theta =\frac{1}{4}\cos\theta+\frac{3}{4}\cos 3\theta
\label{eq:refname2}
\end{equation}

\end{document}